\begin{titlepage}
  \begin{center}
    Министерство образования Республики Беларусь\\[1em]
    Учреждение образования\\
    БЕЛОРУССКИЙ ГОСУДАРСТВЕННЫЙ УНИВЕРСИТЕТ \\
    ИНФОРМАТИКИ И РАДИОЭЛЕКТРОНИКИ\\[1em]

    \begin{minipage}{\textwidth}
      \begin{flushleft}
        \begin{tabular}{ l l }
          Факультет & информационных технологий и управления\\
          Кафедра   & интеллектуальных информационных технологий
        \end{tabular}
      \end{flushleft}
    \end{minipage}\\[1em]

    \begin{flushright}
      \begin{minipage}{0.4\textwidth}
        \textit{К защите допустить:}\\[0.8em]
        Заведующий кафедрой ИИТ\\[0.45em]
        \underline{\hspace*{2.8cm}} В.\,В.~Голенков
      \end{minipage}\\[2.2em]
    \end{flushright}

    %%
    %% ВНИМАНИЕ: на некторых факультетах (ФКП) и кафедрах (ПИКС) слова "ПОЯСНИТЕЛЬНАЯ ЗАПИСКА" предлагается (требуется) оформлять полужирным начертанием. Раскомментируйте нужную для вас строку:
    %%
    %\textbf{ПОЯСНИТЕЛЬНАЯ ЗАПИСКА}\\
    {ПОЯСНИТЕЛЬНАЯ ЗАПИСКА}\\
    {к дипломному проекту}\\
    {на тему:}\\[1em]
    \textbf{\large ВЕБ-ПРИЛОЖЕНИЕ ДЛЯ ОРГАНИЗАЦИИ И АНАЛИЗА РАБОЧЕГО И ЛИЧНОГО ВРЕМЕНИ}\\[1em]


    {БГУИР ДП 1 - 40 03 01 01 093 ПЗ}\\[2em]
    
    \begin{tabular}{ p{0.65\textwidth}p{0.25\textwidth} }
      Студент & С.\,Э.~Родионов \\
      Руководитель & Е.\,Б.~Никитин \\
      Консультанты: &\\
      \hspace*{3ex}\emph{по специальности} & В.\,П.~Ивашенко \\
      \hspace*{3ex}\emph{по экономической части} & Т.\,Л.~Слюсарь \\
      %%
      %% ВНИМАНИЕ: в зависимости от выбранной темы, у вас консультант может быть как по охране труда, так и по:
        % экологической безопасности
        % ресурсосбережению
        % энергосбережению
      %%
      %% Впишите правильную формулировку по необходимости
      Нормоконтролёр & В.\,В.~Захаров\\
      & \\
      Рецензент &
    \end{tabular}
    
    \vfill
    {\normalsize Минск 2017}
  \end{center}
\end{titlepage}